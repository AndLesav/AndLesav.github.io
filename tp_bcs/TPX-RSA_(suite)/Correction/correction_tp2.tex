\documentclass[english,a4paper,11pt,fleqn]{exam}
\usepackage[T1]{fontenc}
\usepackage{babel}
\usepackage{graphicx,algorithmic,algorithm}
\usepackage{color}
\usepackage{listings,url}
\usepackage[left=2cm,right=2cm,top=2cm,bottom=2cm]{geometry}
\usepackage{systeme}

\title{\textbf{TP2 RSA (suite) - Correction}}

\begin{document}
	
	\section*{Exercise 1: Factorization}
	
	Ici on a $n1 = p^2$, donc il est très facile de retrouver la valeur de p, et donc celle de $\varphi(n1)$.
	Attention tout de même, $\varphi(p^2) = p^2 - p\neq (p-1)*(p-1)$.

	À partir de là, on retrouve a valeur \texttt{d = inverse\_mod(e,}  $\varphi(p^2)$), ce qui permet de déchiffrer le message.
	
	\section*{Exercice 2 : Decryption oracle}

	On ne peut pas demander de déchiffrer $c$ directement, mais on peut envoyer $c' = 2^e \times c \bmod n$.\\
	Le serveur répond alors par $m' = c'^d\bmod n = (2^e \times c)^d \bmod n = 2*m\bmod n$. 
	Il suffit alors de multiplier $m'$ par l'inverse de 2 modulo n pour trouver le résultat.

\end{document}
%%% Local Variables:
%%% mode: latex
%%% TeX-master: t
%%% End:
