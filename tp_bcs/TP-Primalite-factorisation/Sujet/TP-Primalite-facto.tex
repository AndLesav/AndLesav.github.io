\documentclass[english,a4paper,11pt]{exam}
\usepackage{lmodern}
\usepackage[utf8]{inputenc}
% \usepackage[T1]{fontenc}
\usepackage{babel}
\usepackage{graphicx,algorithmic,algorithm}

\usepackage{stmaryrd}
\usepackage[dvipsnames]{xcolor}
\usepackage{listings,url}
\usepackage[left=2cm,right=2cm,top=1.5cm,bottom=1.5cm]{geometry}
\usepackage{hyperref}

\lstset{frame=tb,
  language=C,
  aboveskip=2mm,
  belowskip=2mm,
  showstringspaces=false,
  columns=flexible,
  basicstyle={\small\ttfamily},
  numbers=none,
  numberstyle=\color{gray},
  keywordstyle=\color{blue},
  commentstyle=\color{Brown},
  stringstyle=\color{Mulberry},
  breaklines=true,
  breakatwhitespace=true,
  tabsize=3
}

\title{\textbf{Practical Exercises 2 - Primality testing and factorisation}}
\date{\today}

\begin{document}

\maketitle

The goal of these exercises is to familiarize withclassical  arithmetical problems
such as primality testing or factorisation, and their impact on public key
cryptograhy. For implementations, it is advised you do a first draft using
\textsc{SageMath} before translating it into
C with \textsc{Gmp}.

% \noindent A report is expected, as a tar file containing your code as long as your answers to the questions. Do not forget to comment your code, and give details in your answers. Include \textbf{any leads you explored}, even if they did not succeed, and \textbf{as much details as} you can to show your understanding of the subject.

% You shall send your reports in an e-mail, containing "[BCS]" in the subject, to \url{daniel.de-almeida-braga@irisa.fr}. The report is expected within a week after the session.

% You can contact me anytime by mail at
% \url{andrea.lesavourey@irisa.fr}.


\section*{Exercise 1: Primality testing}

One of the main algorithms used to verify whether a given integer \(n\) is a prime
number is \emph{Miller-Rabin} algorithm\footnote{\url{https://en.wikipedia.org/wiki/Miller-Rabin_primality_test}}. It consists in successively drawing integers from
\(\llbracket 2, n-1 \rrbracket \) and checking if they are \emph{witnesses} for the
compositness of \(n\). If no witness is found, then \(n\) is deemed to be prime.

\begin{questions}
  \question Verify experimentally the prime number theorem : compute the average number
  of primality trials required to generate a key for RSA-1024 and compare this value
  to the theoretical one.
  \question Implement the Miller-Rabin primality test.
  \question Under the Generalised Riemann Hypothesis (GRH), the smallest witness
  for a composite \(n\) is known to be less than \(2 \ln n^2\). Deduce from this
  a deterministic primality test, called Miller test.
  \question Compare the performances of your implementations to the ones of
  functions given by \textsc{SageMath} or \textsc{Gmp}.
\end{questions}

\section*{Exercise 2: Pollard's \(p-1\)}

Pollard's \(p-1\) algorithm\footnote{\url{https://en.wikipedia.org/wiki/Pollard's_p_-_1_algorithm}} is a factorisation algorithm based on
the following idea. If \(p \mid n\) then there is
\(a \in \llbracket 2, n-1 \rrbracket\) such that \(a^{p} \equiv 1 \bmod p\).

\begin{questions}
  
  \question Implement Pollard's \(p-1\) algorithm and use it to factorise
  \[
    n = 117827681420271584017432903522327303325344948050665323956545863.
  \]
  
  \question The success of the algorithm depends heavily on the shape of
  the factors of \(n\). After explaining which are such weak factors, write a function
  generating numbers \(n\) for which Pollard's method would not retrieve a factor in
  reasonable time.
\end{questions}

\section*{Exercise 3: Fermat's method}

Fermat's method\footnote{\url{https://en.wikipedia.org/wiki/Fermat's_factorization_method}} tries to factorise an integer \(n\) by writing it as the diffence between
two squares.
\begin{questions}
  \question Implement Fermat method and use it to factorise
  \[
    n = 4433634977317959977189716351978918572296527677331175210881861.
  \]
  \question Can you generate composite integers for which Fermat method fails to
  compute a factor in reasonable time ?
\end{questions}



\end{document}

%%% Local Variables:
%%% mode: latex
%%% TeX-master: t
%%% End:
