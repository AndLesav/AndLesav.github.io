%%%%%%%%%%%%%%%%%%%%%%%%%%%%%%%%%%%%%%%%%%%%%%%%%%%%%%%%%%%%%%%%%%%%%%
% LaTeX Template: Designer's CV
%
% Source: http://www.howtotex.com
% 
% Feel free to distribute this example, but please keep the referral
% to HowToTeX.com
% 
% Date: March 2012
%
% Modified by Lim Lian Tze to support multiple pages using fix provided at
% http://www.howtotex.com/templates/creating-a-designers-cv-in-latex/
% Date: November 2014
%%%%%%%%%%%%%%%%%%%%%%%%%%%%%%%%%%%%%%%%%%%%%%%%%%%%%%%%%%%%%%%%%%%%%%
% How to use writeLaTeX: 
%
% You edit the source code here on the left, and the preview on the
% right shows you the result within a few seconds.
%
% Bookmark this page and share the URL with your co-authors. They can
% edit at the same time!
%
% You can upload figures, bibliographies, custom classes and
% styles using the files menu.
%
% If you're new to LaTeX, the wikibook is a great place to start:
% http://en.wikibooks.org/wiki/LaTeX
%
%%%%%%%%%%%%%%%%%%%%%%%%%%%%%%%%%%%%%%%%%%%%%%%%%%%%%%%%%%%%%%%%%%%%%%

%%%%%%%%%%%%%%%%%%%%%%%%%%%%%%%%%%%%%
% Document properties and packages
%%%%%%%%%%%%%%%%%%%%%%%%%%%%%%%%%%%%%
\documentclass[a4paper,12pt,final]{memoir}

% misc
\renewcommand{\familydefault}{bch}	% font
\pagestyle{empty}					% no pagenumbering
\setlength{\parindent}{0pt}			% no paragraph indentation


% required packages (add your own)
\usepackage{flowfram}										% column layout
\usepackage[top=1cm,left=1cm,right=1cm,bottom=1cm]{geometry}% margins
\usepackage{graphicx}										% figures
\usepackage{url}											% URLs
\usepackage[usenames,dvipsnames]{xcolor}					% color
\usepackage{multicol}										% columns env.
	\setlength{\multicolsep}{0pt}
\usepackage{paralist}										% compact lists
\usepackage{tikz}

\usepackage{ragged2e}

%%%%%%%%%%%%%%%%%%%%%%%%%%%%%%%%%%%%%
% Create column layout
%%%%%%%%%%%%%%%%%%%%%%%%%%%%%%%%%%%%%
% define length commands
\setlength{\vcolumnsep}{\baselineskip}
\setlength{\columnsep}{\vcolumnsep}

% left frame
\newflowframe{0.15\textwidth}{\textheight}{0pt}{0pt}[left]
\newlength{\LeftMainSep}
\setlength{\LeftMainSep}{0.15\textwidth}
\addtolength{\LeftMainSep}{1\columnsep}

% small static frame for the vertical line
\newstaticframe{1.5pt}{\textheight}{\LeftMainSep}{0pt}

% content of the static frame
\begin{staticcontents}{1}
  \hfill
  \tikz{%
    \draw[loosely dotted,color=RoyalBlue,line width=1.5pt,yshift=0]
    (0,0) -- (0,\textheight);}%
  \hfill\mbox{}
\end{staticcontents}

% right frame
\addtolength{\LeftMainSep}{1.5pt}
\addtolength{\LeftMainSep}{1\columnsep}
\newflowframe{0.7\textwidth}{\textheight}{\LeftMainSep}{0pt}[main01]

\linespread{1.25}

%%%%%%%%%%%%%%%%%%%%%%%%%%%%%%%%%%%%% 
% define macros (for convience)
%%%%%%%%%%%%%%%%%%%%%%%%%%%%%%%%%%%%% 
\newcommand{\Sep}{\vspace{1.5em}}
\newcommand{\SmallSep}{\vspace{0.5em}}

\newenvironment{AboutMe}
{\ignorespaces\textbf{\color{RoyalBlue} About me}}
{\Sep\ignorespacesafterend}

\newcommand{\CVSection}[1]
{\Large\textbf{#1}\par
  \SmallSep\normalsize\normalfont}

\newcommand{\CVItem}[1]
{\textbf{\color{RoyalBlue} #1}}


%%%%%%%%%%%%%%%%%%%%%%%%%%%%%%%%%%%%% 
% Begin document
%%%%%%%%%%%%%%%%%%%%%%%%%%%%%%%%%%%%% 
\begin{document}

% Left frame
%%%%%%%%%%%%%%%%%%%% 
% 
% Upload your own photo using the files menu
% \begin{figure}
%   \hfill
%   \includegraphics[width=0.9\columnwidth]{photo.png}
%   \vspace{-7cm}
% \end{figure}
\begin{flushright}\small
  % \url{andrea.lesavourey@irisa.fr}  \\
  % (+61) 401 193 085 \\

  % Andrea Lesavourey \\
\end{flushright}\normalsize
\framebreak


% Right frame
%%%%%%%%%%%%%%%%%%%% 
\Huge\bfseries {\color{RoyalBlue} Andrea Lesavourey}  \\
\normalsize\normalfont

% Education
\CVSection{Academic cursus} 

\CVItem{Post-doctoral fellow}\\
Septembre 2021 - present, Univ Rennes, CNRS, IRISA. \\


\CVItem{PhD Student}\\
October 2017 - June 2021, University of Wollongong, Australia.
\begin{itemize}
\item \textit{Lattices for a post-quantum cryptography}, supervised by Pr Willy Susilo and Dr Thomas Plantard
\end{itemize}

\SmallSep

% \CVItem{Internship supervised by Christophe Negre and Thomas Plantard}\\
% June 2016 - July 2016, Laboratoire DALI, University of Perpignan, France
% \begin{itemize}
% \item Work on Hermite Normal Form and Side Channel Analysis
% \end{itemize}
% \SmallSep


\CVItem{Master 1 in Cryptology}\\
% \justify
2015 - 2016, University of Bordeaux, France

\begin{itemize}
\item Master Research thesis supervised by Christophe Negre and Thomas Plantard : \textit{Randomization in RNS and Leak Resistant Arithmetic} \\
  \underline{Abstract :} The Leak Resistant Arithmetic proposed to randomize an exponentiation procedure in RNS via Montgomery’s multiplication. We study a modification of this approach by not clearing the mask during the procedure in order to save two Montgomery’s multiplications at each loop and improve the level of randomization.

\item Teaching followed : Programming (C), Arithmetic, Complexity Theory, Information Theory, Elliptic Curves (use of Pari/GP), Cryptology, Algorithmic Calculus (use of SageMaths), Introduction to Diophantine Approximations
\end{itemize}
\SmallSep

\CVItem{Master Degree in Pure Mathematics}

2015, University of Bordeaux, France

\begin{itemize}
\item Master Thesis supervised by Pierre Parent : \textit{Th\'{e}or\`{e}me de Chabauty et version effective de Coleman} \\
  \underline{Abstract :} Faltings proved in 1983 that every curve of genus strictly greater than 1 has only a finite number of rational points. Sadly, his proof cannot be made efficient. But Coleman improved the intermediate result of Chabauty (40’s), which use some p-adic argument, to obtain a good bound for the number of rational points in special cases.

\item Teaching followed : Algebraic Geometry, Introduction to p-adic numbers, Computational Number Theory (use of Pari/GP), Group cohomology, Geometry
\end{itemize}

\clearpage
\framebreak
\framebreak 

2012, University of Bordeaux, France
\begin{itemize}
\item Master Thesis supervised by Valentin F\'{e}ray : \textit{Formalisation de la jonglerie et concepts math\'{e}matiques li\'{e}s} \\ 
  \normalsize\normalfont
  \underline{Abstract :} We study siteswaps, which can be defined as one way to juggle and can be described mathematically. In particular, we use 
  different representations of these objects and study them from a combinatorial point of view.
\end{itemize}

\SmallSep
% You'll need these 3 lines at the end of each page!

\CVItem{Master Degree in Teaching of Mathematics}

2012 - 2014, University of Bordeaux, France \vspace{0.2cm}

\hspace{0.4cm} $\bullet$ Agr\'egation de Math\'ematiques, option Probabilit\'es et Statistiques

% \hspace{0.7cm} (use of Scilab)
\vspace{0.2cm}
\SmallSep

\CVItem{Bachelor Degree in Pure Mathematics}\\
2008 - 2011, University of Bordeaux, France

\Sep

\CVSection{Research and academic activities}

\CVItem{Journal papers}
\begin{itemize}
\item  Andrea Lesavourey, Thomas Plantard, and Willy Susilo. ``Short Principal Ideal Problem in multicubic fields''. \textit{Journal of Mathematical Cryptology} 14.1 (2020): 359-392. \url{https://doi.org/10.1515/jmc-2019-0028}
\end{itemize}

\CVItem{Conference papers}

\begin{itemize}
\item Andrea Lesavourey, Thomas Plantard, Willy Susilo:  \textit{On ideal lattices in multicubic fields},  Accepted to Number-Theoretic Methods in  Cryptology (NutMic) 2019, \url{http://nutmic2019.imj-prg.fr/}.
\item Andrea Lesavourey, Christophe Negre, Thomas Plantard:  \textit{Efficient Leak Resistant Modular Exponentiation in RNS.} ARITH 2017: 156-163.
\item Andrea Lesavourey, Christophe Negre, Thomas Plantard:  \textit{Efficient Randomized Regular Modular Exponentiation using Combined Montgomery and Barrett Multiplications.} SECRYPT 2016: 368-375.
\end{itemize}

\CVItem{Current activities}
\begin{itemize}
\item Andrea Lesavourey, Thomas Plantard, Arnaud Sipasseuth,  \textit{Lattices defined by diagonally dominant matrices},  Submitted.
\item Andrea Lesavourey, Thomas Plantard, Willy Susilo,  \textit{On the Short Principal Ideal Problem over some real Kummer fields}, Submitted to the journal \emph{Mathematical Cryptology}.
\end{itemize}  
  \clearpage
\framebreak
\framebreak
\begin{itemize}
\item Andrea Lesavourey, Thomas Plantard, Willy Susilo, \emph{Roots of polynomials in number fields: computation through complex embeddings}, To be submitted.
\end{itemize}


\CVItem{Collaboration visits} \\
June/July 2019, Sorbonne University, LIP6\\
Guest of Jean-Claude Bajard within the MACAO program, \url{https://ssl.informatics.uow.edu.au/MACAO/}. \\
Discussions with Antoine Joux and Fabrice Rouiller on the computation of cube roots in multicubic fields. \\


\CVItem{Organisations} \\
November 2019, MACAO workshop in Wollongong \\
\url{https://ssl.informatics.uow.edu.au/MACAO/workshop_2019.html}. \\

\CVItem{Reviews} \\
ACISP 2020 \\

% Experience
\CVSection{Teaching}

\CVItem{Tutoring in Computer Science}\\
Autumn Session 2020, University of Wollongong\\
Knowledge and Information Engineering (ISIT219)
\SmallSep

\CVItem{Tutoring in Computer Science}\\
Autumn Session 2019, University of Wollongong\\
Problem Solving (CSIT113)
\SmallSep

\CVItem{Highschool Mathematics Teacher}\\
Septembre 2016 - June 2017, Lycée Malherbe of Caen, France\\
Year 10 and Year 11 with specialisation in science
\SmallSep

\CVItem{Tutoring in Mathematics}\\
2015-2016, University of Bordeaux\\
General Mathematics for Bachelor students (50 hours)
\SmallSep

\CVItem{Tutoring in Mathematics}\\
2014-2015, University of Bordeaux\\
General Mathematics for Bachelor students (35 hours)
\SmallSep

\CVItem{Preparatory School Examiner}\\
2014-2015, Camille Jullian Highschool, Bordeaux\\
Oral Examiner for students training to enter Engineering Schools\\
Linear Algebra, Real Analysis (60 hours)
\SmallSep

\CVItem{Private tutoring}\\
From secondary school to bachelor students
\SmallSep
\Sep

\clearpage
\framebreak
\framebreak

\CVSection{References}
\CVItem{University of Wollongong, Australia} 

\textbf{Dr Thomas Plantard} \\
Senior Research Fellow\\
Institute of Cybersecurity and Cryptology \\
School of Computing and Information Technology  \\
\url{thomaspl@uow.edu.au} \\
% (+61) 2 42 21 53 24 \\

\textbf{Pr Willy Susilo} \\
Professor and Head of School\\
Institute of Cybersecurity and Cryptology \\
School of Computing and Information Technology \\
\url{wsusilol@uow.edu.au} \\

\CVItem{University of Perpignan, France} \\
\textbf{Dr Christophe Negre} \\
Associate Professor \\
Digits Architectures Logiciels Informatique\\
\url{christophe.negre@univ-perp.fr} 


%%%%%%%%%%%%%%%%%%%%%%%%%%%%%%%%%%%%%
% End document
%%%%%%%%%%%%%%%%%%%%%%%%%%%%%%%%%%%%%
\end{document}
%%% Local Variables:
%%% mode: latex
%%% TeX-master: t
%%% End:
