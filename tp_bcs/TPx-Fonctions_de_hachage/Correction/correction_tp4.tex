\documentclass[french,a4paper,11pt]{exam}
\usepackage[utf8]{inputenc}
\usepackage[T1]{fontenc}
\usepackage{babel}
\usepackage{graphicx,algorithmic,algorithm}
\usepackage{color}
\usepackage{listings,url}
\usepackage{amsmath,amsfonts}
\usepackage[left=2cm,right=2cm,top=2cm,bottom=2cm]{geometry}

\newcommand{\Zp}[1]{\mathbb{Z}/{#1}\mathbb{Z}}

\title{\textbf{TP4 - Fonctions de hachage}}
\date{19 Octobre 2020}

\begin{document}
	
	\section*{Exercice 1 : Recherche de collision avec l'algorithme de Floyd}
	
	Cf. implémentation sage
	
	\section*{Exercice 2 : SHA-256}

	Cf. implémentation en C (elle n'est pas très joli, je l'ai faite rapidement pour vous fournir une correction).
	
	\section*{Exercice 3 : \textit{Length-extension attack}}

	Je vous renvoie vers un des nombreux tuto disponible en ligne. Celui-ci\footnote{\url{https://crypto.stackexchange.com/questions/3978/understanding-the-length-extension-attack}}\footnote{\url{https://blog.skullsecurity.org/2012/everything-you-need-to-know-about-hash-length-extension-attacks}}, par exemple, offre une description complète de l'attaque.
	
	
\end{document}