\documentclass[fleqn, french,a4paper,11pt]{exam}
\usepackage[T1]{fontenc}
\usepackage{babel}
\usepackage{graphicx,algorithmic,algorithm}
\usepackage{color}
\usepackage{listings,url}
\usepackage{amsmath,amsfonts}
\usepackage[left=2cm,right=2cm,top=2cm,bottom=2cm]{geometry}
\usepackage{hyperref}

\newcommand{\Zp}[1]{\mathbb{Z}/{#1}\mathbb{Z}}

\title{\textbf{TP3 - Correction}}
\date{9 Octobre 2020}

\begin{document}
	
	\maketitle
	
	% \section*{Exercice 0 : Entrainement pour les corps finis  et les calculs modulaires}
	
	
	% \begin{questions}
	
	% \question Il s'agit de calculs basique, sans piège. Pensez bien à réduire les facteur avant d'effectuer l'opération si possible.
	% \begin{parts}
	% 	\part $2\times 8 = 16 \equiv 4 \bmod 12$
	% 	\part $2\times 6 = 12 \equiv 1 \bmod 11$
	% 	\part $9\times 2 \equiv 18 \bmod 20$
	% 	\part $120465 \times  1346546132166510 \equiv 1\times 0 \equiv 0 \bmod 2$
	% 	\part $120465 \times  1346546132166510 \equiv 65\times 10 \equiv 50 \bmod 100$
	% \end{parts}
	% \question Considérons le groupe d'entiers modulo un nombre premier $p = 11$. Ce groupe est souvent noté $\Zp{p}$ et contient les éléments $\{0, 1, \cdots, p-1\}$.
	% \begin{parts}
	% 	\part \begin{itemize}
	% 		\item $g^1 \equiv 2 \bmod 11$
	% 		\item $g^2 \equiv 4 \bmod 11$
	% 		\item $g^3 \equiv 8 \bmod 11$
	% 		\item $g^4 \equiv 5 \bmod 11$
	% 		\item $g^5 \equiv 10 \bmod 11$
	% 		\item $g^6 \equiv 9 \bmod 11$
	% 		\item $g^7 \equiv 7 \bmod 11$
	% 		\item $g^8 \equiv 3 \bmod 11$
	% 		\item $g^9 \equiv 6 \bmod 11$
	% 		\item $g^{10} \equiv 1 \bmod 11$
	% 		\item $g^{11} \equiv 2 \bmod 11$
	% 	\end{itemize}
		
	% 	\part Tous les éléments sont des générateurs, donc à partir de tous les éléments, on peut générer 1 en prenant $a^k\bmod p$ pour un $k$. Alors l'inverse de $a$ est $a^{k-1} \bmod p$.
		
	% 	\part Si $p$ est premier, $g$ est un générateur, donc il permet de retrouver tous les éléments du groupe. La propriété sur l'ordre de $g$ suit immédiatement. 
	% \end{parts}

	% \question Considérons maintenant le cas des groupes dont l'ordre n'est pas premier, par exemple $\Zp{15}$.
	% \begin{parts}
	% 	\part Les générateurs sont \{1, 2, 4, 7, 8, 11, 13, 14\}. Ils sont tous premiers avec $15$.
		
	% 	\part Les autres éléments sont d'ordre 3 ou 5. Ils sont de l'ordre d'un diviseur de 15. 
		
	% 	\part Les éléments générateurs (i.e. ceux qui sont premiers avec 15).
		
	% 	\part Un élément $a$ est d'ordre  $n/pgcd(a, n)$.
	% \end{parts}

	% \question Cela revient à calculer $2021^{2020} \bmod 10$. En réduisant le terme avant d'entreprendre la calcul, on se rend compte que cela revient à calculer $2021^{2020} \equiv 1^{2020} \equiv 1 \bmod 10$.
	
	% \end{questions}


	\section*{Exercice 1: Manipulation de points sur une courbe elliptique}
	
	\begin{questions}
		\question L'implémentation est disponible dans le fichier \texttt{correction.sage}.
		\begin{parts}
			\part -
			\part -
			\part On remarque que l'ordre du point divise celui du groupe. Ici, le groupe est d'ordre 24, donc l'ordre du point généré peut être 1, 2, 4, 6, 12 ou 24 (ici il n'y a pas de point d'ordre 24).
			\part Ici, l'approche la plus simple est de calculer $kP$ en incrémentant $k$ jusqu'à tomber sur l'élément neutre (noté $(0 : 1 : 0)$ dans \texttt{sage}). Le $k$ correspondant est l'ordre du point $P$.\\
			On sait toutefois que l'ordre de $P$ divise celui du groupe. On peut donc se contenter de tester les $k$ divisant 24.
			\part La coordonnée est $x$ correspond à celle d'un point si elle permet de satisfaire l'équation de la courbe sur $\Zp{29}$ pour un $y\in\Zp{29}$. Il faut donc commencer par calculer $x^3-3x + 5$ (toujours sur $\Zp{29}$) et vérifier si le résultat correspond à un carré modulo 29.
			
			Attention, si vous utilisez la méthode \texttt{is\_square} sur un entier "classqiue", le résultat ne sera pas forcement bon. Il faut vérifier si il s'agit d'un carré \textbf{modulo 29}. Un autre méthde pour déterminer s'il s'agit d'un carré est \texttt{legendre\_symbol} ou \texttt{jacobi\_symbol} qui retournent $1$ si il s'agit d'un carré.
			
			\part On peut simplement parcourir les coordonnées en $x$ de 0 à 29,et vérifer s'il existe un point correspondant.
			Si c'est la cas, la coordonnée en $x$ peut correspondre à deux points : $(x, y)$ et $(x, -y)$. N'oubliez pas le point infini.
		\end{parts}
		
		\question Considérons maintenant la courbe définie par $y^2 = x^3 - 3x + 6$ sur les entiers modulo $${p = 51646698564449502183630508998684683453}$$
		Cette courbe comporte $\ell=51646698564449502188925059897707133441$ points (elle est donc d'ordre $\ell$).
		\begin{parts}
			\part L'ordre du point divise l'ordre de la courbe, et $\ell = 743 \times 69511034407065278854542476309161687$, d'où $ord(P) \in \{1, 743, 69511034407065278854542476309161687, \ell\}$.
			\part $P$ est d'ordre 743.
			\part L'ordre du point est petit. Il ne permet donc que de générer peu de points. Dans ce contexte, il est possible de résoudre le logarithme discret, rendant les applications cryptographiques non sécurisées.
			\part Les meilleurs algorithmes connus pour résoudre ECDLP tourne en $\mathcal{O}(\sqrt{n})$, où $n$ représente l'ordre de l'élément. Ici, $n = \ell \approx 2^{125}$, offrant seulement $\sqrt{2^{125}} = 2^{64,5}$ bits de sécurité.

			La sécurité attentu aujourd'hui doit être supérieur à $2^{128}$, c'est pourquoi les courbes couramment utilisées ont un ordre approchant de $2^{256}$ ou plus.
		\end{parts}
		
	\end{questions}
	
	\section*{Exercice 2 : ECDSA}
	
	\begin{questions}
		\question Multiples sources disponibles en ligne. Gardez en tête que \href{https://security.stackexchange.com/questions/93322/difference-between-authentication-integrity-and-data-origin-authentication}{intégrité $\neq$ authenticité}.

		L'idée est que ECDSA est une fonction de signature, permettant d'authentifier un message. C'est à dire qu'un utilisateur ayant une clé $(d, P)$, avec $d$ privé et $P=d\times G$ publique, peut signer un message $m$ a l'aide de sa clé \textbf{privée}. La signature consiste en deux entiers $(r, s)$, considérés publiques.

		Tout utilisateur en possession du message $m$, de la clé publique $P$ et de la signature $(r, s)$ est en mesure de vérifier si la signature est valide.
		
		\question Cf. correction.sage
		
		\question Le nonce est effictivement sensible. Il doit être secret, et différent pour chaque signature effectuée avec la même clé. Toute fuite d'information (même quelques bits) sur le nonce compromet la confidentialité de la clé privée.
	\end{questions}
	
	\section*{Exercice 3 : Attaques sur ECDSA (Bonus)}

	Définissons les notations suivantes :
	\begin{itemize}
		\item $k$ est le nonce
		\item $G$ est le point générateur, publique, utilisé par tout les utilisateur de la courbe $P256$. $G$ est d'ordre $n$ et $P256$ est définie sur $\Zp{p}$.
		\item La bi-clé d'un utilisateur A est notée $(d_A, P_A)$ avec $P_A = d_A\times G$.
		\item Si $P = (x, y)$ est un point sur la courbe, $P_x$ désigne sa coordonnée en $x$.
	\end{itemize}
	
	\begin{questions}
		\question Supposez qu'un utilisateur signe un message et le transmette avec le nonce utilisé durant la signature.
		\begin{parts}
			\part Un attaquant peut retrouver la clé privée :
			Sachant que $s = k^{-1}(h(m) + d\times r) \bmod n$, la seule inconnue est $d$, la clé privée. On peut simplement inverser les opérations une par une pour arriver au résultat (toute les opération si dessous sont effectuées modulo $n$):
			\begin{align*}
				& s = k^{-1}(h(m) + d\times r)\\
				& s\times k = h(m) + d\times r\\
				& s\times k - h(m) = d\times r\\
				& (s\times k - h(m))r^{-1} = d
			\end{align*}
			\part Cf correction.sage
		\end{parts}
		
		\question Cette fois, l'utilisateur a gardé le nonce secret, mais l'a utilisé pour signer plusieurs messages différents (deux en l'occurrence). Ce problème est connu et a causé de nombreux problèmes dans des produits déployés, notamment chez la PS3\footnote{\url{https://media.ccc.de/v/27c3-4087-en-console_hacking_2010}}$^, $\footnote{\url{https://www.bbc.com/news/technology-12116051}}, compromettant entièrement la console.
		\begin{parts}
			\part La valeur de $r$ sera la même car $r = k\times P$. Ici, les message m2\_1 et m2\_3 sont concernés.
			\part On sait que $r_1 = r_2 = r$. $s_1 = k^{-1}(h(m_1) + d\times r)$ et $s_2 = k^{-1}(h(m_2) + d\times r)$.
			En soustrayant les deux valeurs de $s_1$ ety $s_2$, on peut retrouver la valeur du nonce. À partir de là, la même attaque s'applique.

			Les opérations ci-dessous sont toutes effectuées modulo $n$.
			\begin{align*}
				s_1 -s_2 & = k^{-1}(h(m_1) + d\times r) -  k^{-1}(h(m_2) + d\times r) \\
				& = k^{-1}(h(m_1) -  h(m_2))\\
				\Rightarrow k & = (s_1 -s_2)^{-1} \times (h(m_1)-h(m_2))
			\end{align*}
			\part Cf correction.sage
		\end{parts}
		
		
	\end{questions}
	
	
\end{document}
%%% Local Variables:
%%% mode: latex
%%% TeX-master: t
%%% End:
