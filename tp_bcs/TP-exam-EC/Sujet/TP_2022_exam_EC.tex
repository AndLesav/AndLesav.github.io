\documentclass[french,a4paper,11pt]{exam}
\usepackage[T1]{fontenc}
\usepackage{babel}
\usepackage{graphicx,algorithmic,algorithm}
\usepackage{color}
\usepackage{listings,url}
\usepackage{hyperref}
\usepackage{amsmath,amsfonts}
\usepackage[left=2cm,right=2cm,top=2cm,bottom=2cm]{geometry}


\newcommand{\Zp}[1]{\mathbb{Z}/{#1}\mathbb{Z}}

\title{\textbf{TP - ECC et ECDSA}}
\date{\today}

\begin{document}

\maketitle

Le TP d'aujourd'hui est un examen. Un rendu est attendu : une archive contenant tout
votre code. N'oubliez pas de commenter votre code.

Le rendu est à envoyer par mail pour le \textbf{vendredi 7 0ctobre, 23h59}, en incluant "[BCS]" dans l'objet, à l'adresse \url{andrea.lesavourey@irisa.fr}. 

\begin{center}
  \fbox{
    \fbox{
      \parbox{5.5in}{
        L'objectif est d'implanter \textbf{en C} les opérations sur les courbes
        elliptiques nécessaires à la cryptographie.
      }
    }
  }
\end{center}

% \noindent Toutes les données nécessaires sont disponibles sur le drive dans le fichier
% \texttt{tp2-ecdsa\_material.sage}.

\section*{Exercice 1: Manipulation de points sur une courbe elliptique}
Implantez les opérations nécessaires à la cryptographie basée sur les
courbes elliptiques. Vous pouvez par exemple utiliser la librairie
\textsc{Gmp} pour manipuler des entiers de taille arbitraire. 

\vspace{-.5cm}
\section*{}
 Afin de rendre des étapes d'exponentiation plus efficaces, plusieurs
méthodes ont été développées\footnote{\url{https://en.wikipedia.org/wiki/Exponentiation_by_squaring}}\footnote{\url{http://koclab.cs.ucsb.edu/teaching/ecc/eccPapers/Doche-ch09.pdf}}. L'objectif des exercices suivants est d'en explorer deux
d'entre elles.



\section*{Exercice 2: Fenêtre fixe ou méthode \(2^k\)-aire}
La méthode de la fenêtre fixe ou \(2^k\)-aire consiste à calculer à l'avance
certaines puissances de l'élément de base \(g\) qu'on pourra réutiliser durant le
calcul d'éléments de la forme \(g^e\). Adaptez cette méthode à la multiplication d'un
point \(P\) d'une courbe elliptique par un scalaire. \\

\emph{Bonus :} Implantez la version améliorée de cette méthode qu'est la fenêtre
glissante.


\section*{Exercice 3: Forme non adjacente}
La forme non adjacente (NAF) d'un entier \(k\) est une représentation signée équivalente à
la décomposition binaire. Elle permet notamment de remplacer des
additions par des soustractions lors de la multiplication d'un point \(P\) par le
scalaire \(k\).
\begin{questions}
  \question Codez des fonctions permettant de passer d'une décomposition binaire
  d'un entier à sa NAF et vice-versa.
  \question Implantez un algorithme de multiplication d'un point par un scalaire
  qui utilise la NAF.
\end{questions}




\end{document}
%%% Local Variables:
%%% mode: latex
%%% TeX-master: t
%%% End:
